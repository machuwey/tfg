\begin{appendices}
    % Appendix titles modifyers
    \renewcommand{\theHchapter}{A\arabic{chapter}}
    \renewcommand{\theHsection}{A\arabic{section}}
    \renewcommand*{\justifyheading}{\raggedleft}
    \titleformat{\chapter}[display]
        {\vspace{-3cm}
            \normalfont\Huge\bfseries\justifyheading}
        {\color{gray} \fontsize{48}{48}\selectfont \appendixname 
        \hspace{0.1cm} \thechapter}
        {20pt}{\fontsize{48}{48}\selectfont}
    \titleformat{\section}
        {\normalfont\Large\bfseries}{\thesection}{1em}{}
    \titleformat{\subsection}
        {\normalfont\large\bfseries}{\thesubsection}{1em}{}
    \chapter{Manual de Usuario}
        En este anexo se facilitará un guía para el usuario. El propósito de esta guía es explicar al usuario cuáles son las capacidades de la aplicación y su modo de uso.
        \section{Registro}
        Para poder utilizar la aplicación el usuario deberá realizar primero el registro en el sistema. Este se realizará pulsando el botón de ``¿No tienes una cuenta? Regístrate''.
        
        Posteriormente el usuario procederá a rellenar la información que ahí se le pide como nombre, correo electrónico y contraseña que mínimo deberá ser de 6 caracteres.
        En el siguiente paso deberá rellenar datos como nombre de usuario y localización preferida a través de un mapa interactivo, asimismo como aceptar los términos y condiciones.

        Finalmente, para terminar el proceso de registro el usuario deberá confirmar su correo electrónico a través del enlace que le será enviado al buzón del correo electrónico especificado durante el registro.
    \begin{figure}[H]
    \centering
    \begin{minipage}{0.3\textwidth}
        \centering
        \includegraphics[cframe=black 2pt, width=1\linewidth]{images/manual/login.png}
    \end{minipage}
    \begin{minipage}{0.3\textwidth}
        \centering
        \includegraphics[cframe=black 2pt, width=1\linewidth]{images/manual/registro1.png}
    \end{minipage}
    \begin{minipage}{0.3\textwidth}
        \centering
        \includegraphics[cframe=black 2pt, width=1\linewidth]{images/manual/registro2.png}
    \end{minipage}
     \begin{minipage}{0.3\textwidth}
        \centering
        \includegraphics[cframe=black 2pt, width=1\linewidth]{images/manual/mapPicker.png}
    \end{minipage}
    \begin{minipage}{0.3\textwidth}
        \centering
        \includegraphics[cframe=black 2pt, width=1\linewidth]{images/manual/registroConfirmacion.png}
    \end{minipage}
    \caption{Login and Register}
    \label{fig:login_register}
\end{figure}
\section{Iniciar sesión}
Para iniciar sesión simplemente el usuario debe proporcionar las credenciales que uso durante el proceso de registro.
\begin{figure}[H]
        \centering
        \includegraphics[cframe=black 2pt,width=0.3\linewidth]{images/manual/login.png}
        \caption{Iniciar sesión}
        \label{fig:my_label}
\end{figure}
Después de rellenar correo y contraseña el usuario deberá pulsar el botón ``Iniciar sesión'', si las credenciales son correctas la aplicación redirigirá al usuario a la pantalla principal.

\section{Unirse a un grupo}
Estando en la pantalla principal, el usuario puede explorar grupos alrededor de su localización a través de un mapa interactivo.
\begin{figure}[H]
        \centering
        \includegraphics[cframe=black 2pt,width=0.3\linewidth]{images/manual/explorarGruposMapap.png}
        \caption{Iniciar sesión}
        \label{fig:my_label}
\end{figure}
Pinchando sobre el grupo el usuario podrá visualizar una vista detallada de la información del grupo como título, descripción e integrantes. Para unirse a un grupo habrá que pulsar sobre el botón ``Unirse''.

\section{Chatear}
Para acceder al chat de un grupo el usuario deberá dirigirse al listado de grupos pulsando en el primer icono del menú inferior de navegación.
\end{appendices}