\begin{appendices}
    % Appendix titles modifyers
    \renewcommand{\theHchapter}{A\arabic{chapter}}
    \renewcommand{\theHsection}{A\arabic{section}}
    \renewcommand*{\justifyheading}{\raggedleft}
    \titleformat{\chapter}[display]
        {\vspace{-3cm}
            \normalfont\Huge\bfseries\justifyheading}
        {\color{gray} \fontsize{48}{48}\selectfont \appendixname 
        \hspace{0.1cm} \thechapter}
        {20pt}{\fontsize{48}{48}\selectfont}
    \titleformat{\section}
        {\normalfont\Large\bfseries}{\thesection}{1em}{}
    \titleformat{\subsection}
        {\normalfont\large\bfseries}{\thesubsection}{1em}{}
    \chapter{Manual de Usuario}
        En este anexo se facilitará un guía para el usuario. El propósito de esta guía es explicar al usuario cuáles son las capacidades de la aplicación y su modo de uso.
        \section{Registro}
        Para poder utilizar la aplicación el usuario deberá realizar primero el registro en el sistema. Este se realizará pulsando el botón de ``¿No tienes una cuenta? Regístrate''.
        
        Posteriormente el usuario procederá a rellenar la información que ahí se le pide como nombre, correo electrónico y contraseña que mínimo deberá ser de 6 caracteres.
        
    \begin{figure}[H]
    \centering
    \begin{minipage}{0.3\textwidth}
        \centering
        \includegraphics[cframe=black 2pt, width=1\linewidth]{images/manual/login.png}
    \end{minipage}
    \begin{minipage}{0.3\textwidth}
        \centering
        \includegraphics[cframe=black 2pt, width=1\linewidth]{images/manual/registro1.png}
    \end{minipage}
    \begin{minipage}{0.3\textwidth}
        \centering
        \includegraphics[cframe=black 2pt, width=1\linewidth]{images/manual/registro2.png}
    \end{minipage}
     \begin{minipage}{0.3\textwidth}
        \centering
        \includegraphics[cframe=black 2pt, width=1\linewidth]{images/manual/mapPicker.png}
    \end{minipage}
    \begin{minipage}{0.3\textwidth}
        \centering
        \includegraphics[cframe=black 2pt, width=1\linewidth]{images/manual/registroConfirmacion.png}
    \end{minipage}
    \caption{Login y registro}
    \label{fig:login_register}
\end{figure}
En el siguiente paso deberá rellenar datos como nombre de usuario y localización preferida a través de un mapa interactivo, asimismo como aceptar los términos y condiciones.

Finalmente, para terminar el proceso de registro el usuario deberá confirmar su correo electrónico a través del enlace que le será enviado al buzón del correo electrónico especificado durante el registro.
\section{Iniciar sesión}
Para iniciar sesión simplemente el usuario debe proporcionar las credenciales que uso durante el proceso de registro.
\begin{figure}[H]
        \centering
        \includegraphics[cframe=black 2pt,width=0.3\linewidth]{images/manual/login.png}
        \caption{Iniciar sesión}
        \label{fig:login}
\end{figure}
Después de rellenar correo y contraseña el usuario deberá pulsar el botón ``Iniciar sesión'', si las credenciales son correctas la aplicación redirigirá al usuario a la pantalla principal.

\section{Unirse a un grupo}
Estando en la pantalla principal, el usuario puede explorar grupos alrededor de su localización a través de un mapa interactivo.
\begin{figure}[H]
        \centering
        \includegraphics[cframe=black 2pt,width=0.3\linewidth]{images/manual/explorarGruposMapap.png}
        \caption{Explorar grupos}
        \label{fig:explore_groups}
\end{figure}
Pinchando sobre el grupo el usuario podrá visualizar una vista detallada de la información del grupo como título, descripción e integrantes. Para unirse a un grupo habrá que pulsar sobre el botón ``Unirse''.
\begin{figure}[H]
        \centering
        \includegraphics[cframe=black 2pt,width=0.3\linewidth]{images/manual/unirseGrupoVistaDetallada.png}
        \caption{Unirse a un grupo}
        \label{fig:join_group}
\end{figure}
\section{Chatear}
Para acceder al chat de un grupo el usuario deberá dirigirse al listado de grupos pulsando en el primer icono del menú inferior de navegación.
De esta forma podrá visualizar los grupos a los que pertenece ordenados por fecha del último mensaje.
\begin{figure}[H]
        \centering
        \begin{minipage}{0.3\textwidth}
            \centering
            \includegraphics[cframe=black 2pt,width=1\linewidth]{images/manual/listadoGruposChats.png}
        \end{minipage}
        \begin{minipage}{0.3\textwidth}
            \centering
            \includegraphics[cframe=black 2pt,width=1\linewidth]{images/manual/ejemploChat.png}
        \end{minipage}
        \caption{Chat}
        \label{fig:chat}

\end{figure}
Pulsando sobre el chat correspondiente, el usuario podrá acceder a los mensajes del grupo y mandar mensajes por su parte.

\section{Abandonar grupo}
Para abandonar grupo el usuario deberá en la pantalla de detalles del grupo. Pulsar sobre el botón ``Salir'' en la esquina superior derecha. Posteriormente deberá confirmar la acción a través de una ventana emergente.

\begin{figure}[H]
        \centering
        \begin{minipage}{0.3\textwidth}
            \centering
            \includegraphics[cframe=black 2pt,width=1\linewidth]{images/manual/groupInfoAsMember.png}
        \end{minipage}
        \begin{minipage}{0.3\textwidth}
            \centering
            \includegraphics[cframe=black 2pt,width=1\linewidth]{images/manual/confirmarAbandono.png}
        \end{minipage}
        \caption{Abandonar grupo}
        \label{fig:leave_group}
\end{figure}

\section{Ajustes de usuario}
Para configurar distintos aspectos del perfil y de la aplicación el usuario debe pulsar en el tercer icono del menú de navegación. En esta pantalla podrá realizar distintas acciones como: actualizar nombre de usuario, localización preferida, idioma de la aplicación, cambiar de modo visual, actualizar foto de perfil y eliminar cuenta.
\begin{figure}[H]
        \centering
        \includegraphics[cframe=black 2pt,width=0.3\linewidth]{images/manual/confPerfil.png}
        \caption{Configuración de perfil}
        \label{fig:user_settings}
\end{figure}

\section{Crear y modificar grupo}
Para crear un grupo el usuario deberá pulsar el icono de ``+'' en la parte superior derecha del listado de grupos.
\begin{figure}[H]
        \centering
        \begin{minipage}{0.3\textwidth}
            \centering
            \includegraphics[cframe=black 2pt,width=1\linewidth]{images/manual/listadoGruposChats.png}
        \end{minipage}
        \begin{minipage}{0.3\textwidth}
            \centering
            \includegraphics[cframe=black 2pt,width=1\linewidth]{images/manual/crearGrupo.png}
        \end{minipage}
        \caption{Crear grupo}
        \label{fig:create_group}
\end{figure}
Deberá rellenar los datos necesarios como título, descripción, tipo de grupo, límite de integrantes y lugar de celebración.
Para modificar un grupo simplemente el usuario creador deberá pulsar sobre la opción ``Editar'' en la pantalla de información del grupo.

\section{Crear y modificar eventos}
Para crear un evento, el usuario deberá pulsar sobre el botón ``Crear evento'' situado en la pantalla de chat.
\begin{figure}[H]
        \centering
        \begin{minipage}{0.3\textwidth}
            \centering
            \includegraphics[cframe=black 2pt,width=1\linewidth]{images/manual/crearEvento.png}
        \end{minipage}
        \begin{minipage}{0.3\textwidth}
            \centering
            \includegraphics[cframe=black 2pt,width=1\linewidth]{images/manual/crearEventoFormulario.png}
        \end{minipage}
        \caption{Crear evento}
        \label{fig:create_event}
\end{figure}
Deberá rellenar la información requerida así como marca la opción de flexibilidad horaria, que permite especificar a la hora de unirse los integrantes del grupo su disponibilidad horaria para el día del evento.
Las instrucciones para modificar un grupo son análogas.
\section{Unirse y abandonar evento}
El usuario podrá unirse a un evento pulsando el botón de ``Unirse a evento'' en caso de la existencia de este.
\begin{figure}[H]
        \centering
        \includegraphics[cframe=black 2pt,width=0.3\linewidth]{images/manual/unirseAEventoBotón.png}
        \caption{Botón unirse a evento}
        \label{fig:join_event_button}
\end{figure}
Será mostrado una pantalla con los detalles del evento y si se da el caso podrá especificar su disponibilidad horaria a través de un selector de rango.
\begin{figure}[H]
        \centering
        \includegraphics[cframe=black 2pt,width=0.3\linewidth]{images/manual/unirseAEvento.png}
        \caption{Unirse a evento}
        \label{fig:join_event}
\end{figure}
Para abandonar el evento tiene que realizar las mismas acciones y en vez de ``Unirse'' se mostrará la opción ``Desapuntarse''.
\begin{figure}[H]
        \centering
        \begin{minipage}{0.3\textwidth}
            \centering
            \includegraphics[cframe=black 2pt,width=1\linewidth]{images/manual/abandonarEventoBotón.png}
        \end{minipage}
        \begin{minipage}{0.3\textwidth}
            \centering
            \includegraphics[cframe=black 2pt,width=1\linewidth]{images/manual/desapuntarse.png}
        \end{minipage}
        \caption{Despuntarse del evento}
        \label{fig:leave_event}
\end{figure}
\end{appendices}