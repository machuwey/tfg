\chapter{Conclusiones y Líneas Futuras}
    \section{Conclusiones}
        Conforme a la planificación estipulada al comienzo del proyecto, durante los últimos meses hemos logrado desarrollar la versión inicial de nuestra aplicación móvil. Esta primera versión, aunque con una interfaz sencilla y funcionalidades básicas, podría clasificarse como el Producto Mínimo Viable\cite{REF6} de nuestra aplicación. Consideramos este punto de partida como un logro significativo, ya que la aplicación cumple con criterios esenciales de usabilidad, fiabilidad, funcionalidad y diseño, elementos suficientes para introducirla en el mercado.

        \section{Líneas futuras}
            En esta sección se presentarán las posibles mejoras y extensiones, que podemos implementar en nuestra aplicación, con el objetivo de mejorar la experiencia de usuario y aumentar el alcance de nuestra plataforma.
  
        \subsection{Mejoras en el chat}
            Vemos un amplio margen de mejora en la funcionalidad del chat de nuestra aplicación. Entre las potenciales mejoras que estamos considerando, destaca la posibilidad de compartir imágenes y vídeos. Esta característica, común en muchas plataformas de chat, enriquecería la comunicación entre los usuarios, permitiendo un intercambio de información más completo y dinámico.
            
        \subsection{Sistema de reporte}
            Dado que los usuarios pueden acceder al contenido creado por otros, es crucial implementar un sistema que permita gestionar y moderar este contenido. En este sentido, una posible solución sería la incorporación de un sistema de reportes. En este sistema, los usuarios podrían reportar entidades, ya sean grupos o individuos, que violen las normas de la comunidad. Tras un reporte, se activaría un protocolo, ya sea humano o automatizado, para llevar a cabo las acciones pertinentes, asegurando así un ambiente seguro y respetuoso para todos los usuarios.

        \subsection{Mejoras en la planificación de actividades}
            Desarrollar más a fondo este aspecto podría añadir un gran valor a nuestra aplicación. Mejorar esta herramienta permitiría a los grupos organizar sus actividades de manera más eficiente, por ejemplo añadir la posibilidad de además de poder sugerir la ventana temporal de disponibilidad, poder sugerir días. Asimismo poder asignar a un mismo día varias ventanas temporales. Siguiendo un enfoque similar al de la aplicación \textit{LettuceMeet}\cite{REF7}, que permite a los usuarios agregar su disponibilidad en una cuadrícula para identificar el intervalo de tiempo que mejor se superpone con la disponibilidad de todos los miembros.

        \subsection{Aplicación multiplataforma}
            Para maximizar su accesibilidad y alcance, nuestra aplicación debe ser multiplataforma, es decir, compatible con diversos sistemas operativos y dispositivos. En este sentido, es esencial considerar desarrollar la aplicación para Android, dada su predominancia en el mercado global de dispositivos móviles.

        \subsection{Internacionalización}
            Si bien la aplicación actualmente soporta español e inglés, es fundamental para futura expansión global considerar la incorporación de otros idiomas. Aunque estos dos idiomas cubren una gran parte de la población mundial, una verdadera internacionalización implica la inclusión de una amplia gama de idiomas y regionalismos. Por ejemplo, podríamos considerar la adición de francés, alemán, italiano, portugués, chino, japonés, árabe, entre otros.
