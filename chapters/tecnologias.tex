\chapter{Introducción}
    \section{Swift}
    Swift\cite{REF5} es un lenguaje de multiparadigma de código abierto lanzado en 2014 por Apple, orientado al desarrollo de aplicaciones para iOS y macOS. Swift esta fuertemente enfocado en facilitar la experiencia al desarrollador a través de seguridad, eficiencia y expresividad del lenguaje.
    Por ejemplo, la memoria se gestiona automáticamente mediante un recuento de referencias preciso y determinista, manteniendo al mínimo el uso de la memoria sin la sobrecarga de la recolección de basura. Ofrece la posibilidad de escribir código concurrente con simples palabras clave incorporadas que definen el comportamiento asíncrono, haciendo que el código sea más legible y menos propenso a errores.
    

    \begin{figure}[H]
        \centering
        \includegraphics[width=.5\linewidth]{images/Swift_logo.svg.png}
        \caption{Logo Swift}
        \label{fig:logoSwift}
    \end{figure}

    Swift se propone como un reemplazo para los lenguajes basados en C (C, C++ y Objective-C). Como tal, Swift es comparable a esos lenguajes en rendimiento para la mayoría de las tareas. Aún así, ofrece interoperabilidad con Objective-C

        \section{Firebase}
    Firebase es una plataforma de desarrollo de aplicaciones creada por Google, diseñada para ayudar a los desarrolladores a crear, mejorar y hacer crecer sus aplicaciones de manera más eficiente. Proporciona una variedad de herramientas y servicios que cubren varias áreas del desarrollo de aplicaciones, incluyendo alojamiento en la nube, bases de datos en tiempo real, autenticación de usuarios, análisis de aplicaciones, pruebas A/B, etc.

    \begin{figure}[H]
        \centering
        \includegraphics[width=.5\linewidth]{images/1280px-Firebase_Logo.svg.png}
        \caption{Logo Firebase}
        \label{fig:Logo Firebase}
    \end{figure}

    Una de las características más notables de Firebase es su base de datos en tiempo real, que permite a los desarrolladores almacenar y sincronizar datos entre usuarios en tiempo real. Esto es especialmente útil para aplicaciones que requieren actualizaciones en tiempo real, como juegos en línea, aplicaciones de chat, y aplicaciones colaborativas.

    Además, Firebase también incluye una potente autenticación de usuarios\cite{REF8}, permitiendo a los desarrolladores implementar sistemas de inicio de sesión seguros y personalizados sin tener que construirlos desde cero. Firebase admite varios métodos de autenticación, incluyendo correo electrónico/contraseña, autenticación telefónica, Google, Facebook, Twitter, y GitHub.

    Otra de las características a mencionar es Firebase Analytics, que ofrece a los desarrolladores una visión detallada del uso y el comportamiento de sus aplicaciones. Esta herramienta puede ayudar a los desarrolladores a entender cómo los usuarios interactúan con sus aplicaciones, lo que puede ser útil para la toma de decisiones y la optimización del rendimiento de la aplicación.
    
    \section{XCode}

    Xcode es un entorno de desarrollo integrado (IDE) desarrollado por Apple para el desarrollo de software en Mac. Es la principal herramienta utilizada para desarrollar software para iOS, macOS, watchOS y tvOS. Xcode proporciona una amplia variedad de herramientas para el desarrollo de software, incluyendo compiladores para el lenguaje de programación Swift y Objective-C, herramientas para la construcción de interfaces de usuario, y utilidades para la gestión de dispositivos, el despliegue de aplicaciones y la gestión de la App Store.

    Además de sus capacidades de programación, Xcode también ofrece una variedad de características para facilitar el desarrollo de software. Por ejemplo, el depurador de Xcode permite a los desarrolladores examinar el estado de un programa en cualquier punto de su ejecución, mientras que su perfilador ayuda a identificar cuellos de botella en el rendimiento del software. Xcode también incluye un editor de interfaces de usuario gráficas, que permite a los desarrolladores diseñar las interfaces de sus aplicaciones de manera visual.

    Xcode también se integra con otras herramientas y servicios de Apple. Por ejemplo, se puede utilizar junto con el sistema de gestión de versiones de Git para el control de versiones, y con el servicio de integración continua de Apple, Xcode Server, para la automatización de las pruebas y la construcción de aplicaciones.

    En resumen, Xcode es una herramienta integral para el desarrollo de software en las plataformas de Apple, proporcionando todo lo necesario para diseñar, desarrollar, probar y desplegar aplicaciones.
    
    \section{Visual Paradigm}
    Visual Paradigm es una aplicación de software diseñada para equipos de desarrollo de software para modelar sistemas de información de negocios y gestionar procesos de desarrollo. Además del soporte para modelado, esta tecnología proporciona generación de informes y capacidades de ingeniería de código, incluyendo generación de código. Esta tecnología puede revertir diagramas de ingeniería a partir del código y proporcionar ingeniería de ida y vuelta para varios lenguajes de programación. Visual Paradigm cuenta con diagramas de Lenguaje Unificado de Modelado (UML), Diagrama de Relaciones de Entidades (ERD), y diagramas de Mapeo de Objetos Relacionales (ORMD) esenciales en el diseño de sistemas y bases de datos.
    \begin{figure}[H]
        \centering
        \includegraphics[width=.5\linewidth]{images/visual-paradigm.png}
        \caption{Logo Visual Paradigm}
        \label{fig:Logo Visual Paradigm}
    \end{figure}