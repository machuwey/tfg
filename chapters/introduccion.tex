\chapter{Introducción}
\section{Motivación}
Hoy en día, las aplicaciones móviles se han vuelto herramientas 
esenciales en nuestras vidas cotidianas, ya que facilitan la 
comunicación, la búsqueda de información, entre otras funciones. 
En este contexto, la colaboración y las actividades en grupo son 
aspectos cada vez más relevantes, y las personas buscan aplicaciones 
que les permitan conectarse con otros usuarios que compartan sus 
intereses y trabajar juntos en proyectos específicos.

Por otro lado, las redes sociales y las aplicaciones de comunicación 
han experimentado un crecimiento significativo en la última década, 
pero aún hay oportunidades para mejorar y personalizar la experiencia 
del usuario en función de sus intereses y objetivos. Es aquí donde 
surge la necesidad de nuestra aplicación móvil, cuya finalidad es 
facilitar la conexión entre personas con intereses afines y promover 
la realización de actividades en grupo.

\section{Objetivos}
El objetivo de este Trabajo de Fin de Grado es desarrollar una 
aplicación móvil para iOS que permita a los usuarios 
encontrar personas con intereses afines y llevar a cabo actividades 
en grupo de manera efectiva y adaptada a sus necesidades. Algunas 
de las características de la app son: registro de usuarios, creación 
de grupos, búsqueda de grupos, chat de grupo, entre otras.

\begin{comment}
Para satisfacer las necesidades de los usuarios, se ha diseñado una 
interfaz amigable y funcional que facilite la interacción y el uso de 
la aplicación.
\end{comment}
\section{Estructura del documento}
\begin{itemize}
\item Capítulo 1: Introducción. En este primer capítulo se presenta la 
motivación, los objetivos y la estructura del documento.
\item Capítulo 2: Estado del arte. En el segundo capítulo, se 
realiza un análisis de las aplicaciones móviles existentes en el 
ámbito de la colaboración y actividades en grupo.
\item Capítulo 3: Tecnologías utilizadas. En el tercer capítulo, 
se describen las tecnologías empleadas en el desarrollo de la aplicación.
\item Capítulo 4: Análisis del sistema. En el cuarto capítulo, 
se detallan las funcionalidades de la aplicación y cómo 
está diseñado el sistema.
\item Capítulo 5: Diseño del sistema. En el quinto capítulo, se elabora sobre las decisiones que se han tomado a la hora de arquitectura y diseño de la aplicación.
\item Capítulo 6: Conclusión. En el sexto capítulo, se realiza 
una valoración del proyecto y se comentan las posibles líneas futuras.
\item Capítulo 7: Bibliografía. En el séptimo capítulo, se enumeran 
las fuentes consultadas para el desarrollo de la aplicación. 

\end{itemize}