\begin{table}[h]
    \centering
    \begin{tabular}{|m{3cm}|m{11cm}|}
        \hline
        \rowcolor{blue!20} Caso de uso & Crear evento (CU08) \\
        \hline
        Actor & Usuario \\
        \hline
        \rowcolor{blue!20} Descripción & El usuario podrá crear un nuevo evento dentro de un grupo proporcionando datos como lugar, fecha, hora y descripción del evento. \\
        \hline
        Pre-condiciones & El usuario tiene la sesión iniciada y es propietario del grupo en el que quiere crear el evento \\
        \hline
        \rowcolor{blue!20} Post-condiciones & Un nuevo evento queda registrado en la base de datos \\
        \hline
        Flujo normal & 
            \begin{enumerate}[noitemsep]
            \item El usuario selecciona el grupo en el que quiere crear el evento desde el listado de grupos
            \item El usuario pulsa en el botón \enquote{Crear evento}
            \item El sistema presenta un formulario
            \item El usuario rellena el formulario con los datos del nuevo evento
            \item El usuario pulsa en el botón \enquote{Crear}
            \item El sistema comprueba los campos
            \item El sistema registra el evento en la base de datos
            \item El usuario es redirigido al chat del grupo
            \end{enumerate}
         \\
         \hline
        Flujo alternativo & 
        \begin{enumerate}[noitemsep]
            \item[4.1] Los datos introducidos son inválidos
            \begin{enumerate}[noitemsep]
                \item[4.1.1] El sistema muestra un mensaje de error
                \item[4.1.2] El usuario corrige los datos
            \end{enumerate}
        \end{enumerate} \\
        \hline
    \end{tabular}
    \caption{Especificaciones del caso de uso: Crear evento (CU08)}
\end{table}
