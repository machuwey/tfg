\section{Requisitos funcionales}
        \begin{itemize}
            \item \textbf{RF0. Registro}: El usuario se dará de alta en la aplicación aportando los datos necesarios como son el correo electrónico, contraseña, nombre de usuario y localización preferida.
            \item \textbf{RF1. Inicio de sesión}: El usuario, con el correo electrónico y contraseña creada durante el registro, podrá iniciar sesión en la aplicación.
            \item \textbf{RF2. Restablecer contraseña}: El usuario, proporcionando su correo electrónico, podrá solicitar restablecer su contraseña, a través de un correo que recibe en su correo electrónico.
            \item \textbf{RF3. Cerrar sesión}: El usuario podrá cerrar sesión en la aplicación.
            \item \textbf{RF4. Editar perfil}: El usuario tendrá la posibilidad de modificar datos como el nombre de usuario y la localización preferida.
            \item \textbf{RF5. Crear grupo}: El usuario podrá crear un nuevo grupo de actividad proporcionando datos como lugar, tipo de grupo, límite de integrantes y descripción del grupo.
            \item \textbf{RF6. Explorar grupos}: El usuario podrá a través de un mapa, explorar los grupos disponibles cercanos, pudiendo ver sobre estos la información básica como es el lugar, número actual de integrante y descripción del grupo.
            \item \textbf{RF7. Unirse a un grupo}: El usuario podrá unirse a un grupo siempre y cuando haya espacio disponible.
            \item \textbf{RF8. Salir del grupo}: El usuario podrá salir del grupo en el momento que sea, siempre y cuando no sea el último miembro.
            \item \textbf{RF9. Visualizar listado de grupos}: El usuario podrá visualizar de un vistazo rápido los grupos a los que pertenece y ver el número de mensajes no leídos correspondiente al chat del grupo.
            \item \textbf{RF10. Chat grupal}: El usuario, podrá usar el correspondiente chat de grupo para mandar mensajes en tiempo real en el chat grupal asociado al grupo.
            \item \textbf{RF11. Crear evento}: El usuario propietario del grupo, podrá crear y planificar una actividad a la que los demás miembros del grupo podrán unirse.
            \item \textbf{RF12. Unirse a evento}: El usuario podrá unirse a un evento creado por el propietario del grupo.
        \end{itemize}

    \section{Requisitos no funcionales}
        \begin{itemize}
        \item \textbf{RNF0. Confirmación de correo electrónico}: Como medida de seguridad, se requerirá que los usuarios confirmen su dirección de correo electrónico antes de poder acceder a la aplicación. Esto ayudará a prevenir el uso indebido y garantizará que los correos enviados por la aplicación lleguen a los usuarios correctos.
        \item \textbf{RNF1. Límite de creación de grupos}: Para evitar la sobrecarga del sistema, cada usuario solo podrá crear un máximo de dos grupos.
        \item \textbf{RNF2. Límite de grupos}: Un usuario solo puede ser miembro de 4 grupos simultáneamente.
        \item \textbf{RNF3. Rendimiento}: La aplicación debe ser capaz de manejar un gran número de usuarios simultáneamente sin degradar la calidad del servicio.
        \item \textbf{RNF4. Seguridad}: Los datos del usuario, incluyendo información personal y contraseñas, deben ser almacenados de manera segura para prevenir el acceso no autorizado.
        \item \textbf{RNF5. Internacionalización}: La aplicación deberá estar disponible en inglés y español para facilitar su uso a un público más amplio.
        \item \textbf{RNF6. Modo claro/oscuro}: La aplicación deberá permitir a los usuarios cambiar entre un modo claro y oscuro, mejorando así la experiencia del usuario y permitiéndole adaptar la interfaz a sus preferencias personales.
        \item \textbf{RNF7. Eliminación de cuenta}: El usuario podrá eliminar su cuenta en cualquier momento, eliminando así todos los datos asociados a la misma.

    \end{itemize}