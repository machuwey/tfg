\chapter{Estado del arte}
\section{Introducción}
En este capítulo, se abordará el estado del arte en
cuanto a aplicaciones móviles para la colaboración y
actividades en grupo, destacando las características y
funcionalidades más relevantes de las soluciones existentes.
Además, se discutirán las áreas donde estas aplicaciones
presentan oportunidades de mejora y cómo nuestra propuesta
busca abordar dichas brechas.

\section{Estado del arte}
\subsection{Aplicaciones de colaboración y actividades en grupo}
Existen aplicaciones móviles enfocadas en la colaboración y actividades en grupo en distintos ámbitos, como la productividad, el aprendizaje y el entretenimiento. Algunos ejemplos incluyen Trello, Slack, Microsoft Teams, Meetup, entre otros. Estas aplicaciones ofrecen funcionalidades como la creación de grupos, asignación de tareas, comunicación en tiempo real y seguimiento del progreso.

\subsection{Tendencias en redes sociales y aplicaciones de comunicación}
Las redes sociales y aplicaciones de comunicación también han evolucionado para incluir características que fomentan la colaboración y actividades en grupo. Por ejemplo, Facebook y LinkedIn permiten la creación de grupos temáticos, mientras que aplicaciones como WhatsApp y Telegram ofrecen chats grupales.

\subsection{Búsqueda y conexión de personas con intereses afines}
A pesar de la existencia de aplicaciones que fomentan la colaboración y actividades en grupo, hay pocas soluciones específicamente enfocadas en conectar a personas con intereses afines para llevar a cabo proyectos o actividades conjuntas. Algunas aplicaciones, como Meetup o Bumble Bizz, se acercan a esta propuesta, pero aún hay oportunidades de mejora y personalización en función de los intereses de los usuarios y la naturaleza de las actividades.

\section{Conclusión}
El estado del arte en cuanto a aplicaciones móviles para la colaboración y actividades en grupo muestra una variedad de soluciones existentes, pero también revela oportunidades de mejora y personalización. Nuestra propuesta busca abordar estas brechas mediante el desarrollo de una aplicación que permita a los usuarios encontrar personas con intereses afines y llevar a cabo actividades en grupo de manera efectiva y adaptada a sus necesidades.
