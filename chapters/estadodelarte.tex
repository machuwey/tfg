\chapter{Estado del arte}
\section{Introducción}
En este capítulo, se abordará el estado del arte en
cuanto a aplicaciones móviles para la colaboración y
actividades en grupo, destacando las características y
funcionalidades más relevantes de las soluciones existentes.
Además, se discutirán las áreas donde estas aplicaciones
presentan oportunidades de mejora y cómo nuestra propuesta
busca abordar dichas brechas.

\section{Estado del arte}
\subsection{Meetup}
Meetup es una plataforma en línea que se fundó en junio de 2002 por Scott Heiferman 
y cuatro co-fundadores. La idea de Meetup surgió de la experiencia de Heiferman de 
conocer a sus vecinos en la ciudad de Nueva York por primera vez después de los 
ataques del 11 de septiembre en las Torres Gemelas. La plataforma se creó con el 
objetivo de fomentar la comunidad y la interacción en persona.

Entre las características de Meetup podemos encontrar las siguientes:

\begin{itemize}
  \item \textbf{Organización de grupos:} Los usuarios pueden formar y unirse a grupos basados en intereses comunes.
  \item \textbf{Planificación de eventos:} Los organizadores pueden planificar eventos dentro de su grupo, estableciendo la fecha, la hora, el lugar y los detalles del evento.
  \item \textbf{RSVP para eventos:} Los usuarios pueden confirmar su asistencia a los eventos y ver quién más planea asistir.
  \item \textbf{Comunicación:} Los organizadores y los miembros del grupo pueden comunicarse a través de mensajes dentro de la plataforma, permitiendo la coordinación y el intercambio de información.
  \item \textbf{Categorización de grupos:} Los grupos pueden ser categorizados en base a una variedad de intereses y temas, facilitando a los usuarios encontrar grupos que coincidan con sus intereses.
  \item \textbf{Calendario de eventos:} Los eventos futuros se muestran en un calendario, permitiendo a los usuarios ver los próximos eventos de un vistazo.
  \item \textbf{Eventos locales y virtuales:} Meetup permite la organización de eventos tanto en persona como virtuales, ofreciendo flexibilidad a los usuarios y organizadores.
  \item \textbf{Fotos y comentarios de eventos:} Después de un evento, los asistentes pueden publicar fotos y comentarios, proporcionando un registro de la actividad del grupo y permitiendo a los miembros interactuar después del evento.
\end{itemize}
\begin{figure}[H]
        \centering
        \includegraphics[width=.5\linewidth]{images/Meetup_Logo.png}
        \caption{Logo Swift}
        \label{fig:my_label}
    \end{figure}

\section{Conclusión}

Meetup se centra en fomentar las interacciones en el mundo real a 
través de experiencias compartidas. Sin embargo, su enfoque es más 
general y no está diseñado específicamente para la colaboración en 
proyectos concretos o para la formación de grupos con el objetivo de 
realizar actividades específicas a largo plazo. Además, a pesar de su 
popularidad, Meetup no está exento de críticas, particularmente en lo 
que respecta a la experiencia del usuario y a las tarifas de suscripción 
para los organizadores de grupos.

Nuestra propuesta busca abordar estas brechas al proporcionar una 
plataforma más personalizada y simple, orientada a la colaboración en proyectos y 
actividades específicas. A diferencia de Meetup, nuestra aplicación se 
enfocará en la formación de grupos con objetivos a largo plazo y proporcionará 
herramientas robustas para la gestión de proyectos y la comunicación en grupo.
