\documentclass{book}
\usepackage{array}
\usepackage[table]{xcolor}
\usepackage{lipsum}
\usepackage{booktabs}
\usepackage{indentfirst}
\setlength{\parskip}{12pt}
\setlength{\parindent}{12pt}
\begin{document}
\begingroup
\chapter{Resumen}
En la actualidad, existe una gran cantidad de aplicaciones 
móviles diseñadas para facilitar las relaciones entre personas. 
Estas aplicaciones, conocidas como redes sociales, han adquirido 
una enorme popularidad entre adultos y jóvenes en la última década, 
transformando la forma en que las personas interactúan entre sí.

Por un lado, encontramos redes sociales centradas en publicaciones, 
como Facebook, Twitter e Instagram, donde los usuarios asumen roles 
de consumidores y productores. Por otro lado, existen redes sociales 
orientadas a la comunicación, como WhatsApp y Telegram, en las que 
podemos identificar distintos tipos de interacción: usuario a usuario, 
usuario a grupo y grupo a grupo.

No obstante, hay pocas aplicaciones dedicadas a fomentar la colaboración 
y actividades en grupo para alcanzar objetivos comunes entre las personas, 
donde los usuarios puedan encontrar a otros con intereses similares y 
realizar actividades conjuntas, como por ejemplo, formar una banda de música.

El propósito de este trabajo de fin de grado es desarrollar una 
aplicación móvil que permita a los usuarios encontrar personas con 
intereses afines y llevar a cabo actividades en grupo. Para ello, 
se ha creado una aplicación móvil para iOS utilizando desarrollo 
nativo y el lenguaje de programación Swift. Algunas de las características 
de la app incluyen: registro de usuarios, creación de grupos, búsqueda de 
grupos, chat de grupo, entre otras.

\textbf{\\\large Palabras clave: }

\chapter{Abstract}
\lipsum[2-4]

\textbf{\\\large Keywords: }
\endgroup

\begin{table}[ht]
    \centering
    \begin{tabular}{|m{3cm}|m{10cm}|}
        \hline
        \rowcolor{blue!20} Primary Actors        & Usuario                   \\
        \hline
        Level                                    & N/A                       \\
        \hline
        \rowcolor{blue!20} Complexity            & Low                       \\
        \hline
        Use Case Status                          & N/A                       \\
        \hline
        \rowcolor{blue!20} Implementation Status & N/A                       \\
        \hline
        Preconditions                            & N/A                       \\
        \hline
        \rowcolor{blue!20} Post-conditions       & N/A                       \\
        \hline
        Author                                   & Matvey Dergunov Bushmanov \\
        \hline
        \rowcolor{blue!20} Assumptions           & N/A                       \\
        \hline
    \end{tabular}
    \caption{Abandonar grupo Requirements Spec}
\end{table}

\tableofcontents

\end{document}